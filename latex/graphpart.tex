\documentclass[11pt]{article}

\usepackage{amssymb}
\usepackage{amsmath}
\usepackage{graphicx}
\usepackage{cite}
\usepackage{algorithmic}
\usepackage{algorithm}
\usepackage{todonotes}
\usepackage{url}
\usepackage{tikz}
\usetikzlibrary{arrows}

\usepackage{listings}
 \lstset{
            language=Matlab,                                % choose the language of the code
    %       basicstyle=10pt,                                % the size of the fonts that are used for the code
            numbers=left,                                   % where to put the line-numbers
            numberstyle=\footnotesize,                      % the size of the fonts that are used for the line-numbers
            stepnumber=1,                                           % the step between two line-numbers. If it's 1 each line will be numbered
            numbersep=5pt,                                  % how far the line-numbers are from the code
    %       backgroundcolor=\color{white},          % choose the background color. You must add \usepackage{color}
            showspaces=false,                               % show spaces adding particular underscores
            showstringspaces=false,                         % underline spaces within strings
            showtabs=false,                                         % show tabs within strings adding particular underscores
    %       frame=single,                                           % adds a frame around the code
    %       tabsize=2,                                              % sets default tabsize to 2 spaces
    %       captionpos=b,                                           % sets the caption-position to bottom
            breaklines=true,                                        % sets automatic line breaking
            breakatwhitespace=false,                        % sets if automatic breaks should only happen at whitespace
            escapeinside={\%*}{*)}                          % if you want to add a comment within your code
}

\setlength{\paperwidth}{8.5in}
\setlength{\paperheight}{11in}
\setlength{\voffset}{-0.2in}
\setlength{\topmargin}{0in}
\setlength{\headheight}{0in}
\setlength{\headsep}{0in}
\setlength{\footskip}{30pt}
\setlength{\textheight}{9.25in}
\setlength{\hoffset}{0in}
\setlength{\oddsidemargin}{0in}
\setlength{\textwidth}{6.5in}
\setlength{\parindent}{0in}
\setlength{\parskip}{9pt}

\newcommand{\ben}{\begin{enumerate}}
\newcommand{\een}{\end{enumerate}}

\DeclareGraphicsRule{.JPG}{eps}{*}{`jpeg2ps #1}

\title{Paper Title}
\author{Author1, Author2}
\date{}
\begin{document}
\maketitle
%\tableofcontents
\abstract{Example Abstract}
\section{Introduction}



Example vector 

\begin{align}x_0^T = \bordermatrix{ 
& & & & s&  \cr
& \infty, & \infty, & \cdots & 0,& \cdots &\infty \cr
} \end{align}

Example matrix

\begin{align*}
\begin{bmatrix}
0 & w_{12} & w_{13}  \cr
w_{21} & 0 &  \infty   \cr
\infty& w_{32} & 0   
\end{bmatrix}\end{align*}

Example graph

\begin{align*}
\begin{tikzpicture}[->,>=stealth',shorten >=1pt,auto,node distance=3cm,
thick,main node/.style={circle,draw,font=\sffamily\Large\bfseries}]
  \node[main node] (1) {1};
  \node[main node] (2) [below left of=1] {2};
  \node[main node] (3) [below right of=1] {3};
  \path[every node/.style={font=\sffamily\small}]
    (1) edge node [left] {$w_{13}$} (3)
        edge [bend left] node[left] {$w_{12}$} (2)
    (2) edge [bend left] node {$w_{21}$} (1)
    (3) edge node {$w_{32}$} (2);
\end{tikzpicture}\end{align*}

Example Citation
\cite{Stanton:2012:SGP:2339530.2339722}

%example figure
%\begin{figure}
%\label{rpls}
%\center \includegraphics*[width=5in]{mat.png}
%\end{figure}

\bibliographystyle{plain}
\bibliography{bib}


\end{document}