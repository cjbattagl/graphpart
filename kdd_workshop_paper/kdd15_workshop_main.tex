% This is "sig-alternate.tex" V2.0 May 2012
% This file should be compiled with V2.5 of "sig-alternate.cls" May 2012
%
% This example file demonstrates the use of the 'sig-alternate.cls'
% V2.5 LaTeX2e document class file. It is for those submitting
% articles to ACM Conference Proceedings WHO DO NOT WISH TO
% STRICTLY ADHERE TO THE SIGS (PUBS-BOARD-ENDORSED) STYLE.
% The 'sig-alternate.cls' file will produce a similar-looking,
% albeit, 'tighter' paper resulting in, invariably, fewer pages.
%
% ----------------------------------------------------------------------------------------------------------------




%%%%%% TO   DOO
% Intro - GraphCHI GraphLab? Related work, defend distributed approach
% Add comments on 
% Why is this better than existing approaches i.e in GraphLab...

% PARMETIS !!!!
% Build table of weak/strong scaling time plots
% Build table of quality plots

%Robert:
% Motivating Applications
% Related Work


\documentclass{sig-alternate}

\usepackage{flushend}
\usepackage{vuduc-stdpkgs}
\usepackage{vuduc-typography}

\usepackage{xspace}
\usepackage{url}
%\usepackage{color}
\usepackage{algorithm2e}
\usepackage{amsmath}

\usepackage[table]{xcolor}
\usepackage{booktabs}

\usepackage[binary-units]{siunitx} % for nice SI units
\DeclareSIUnit\hours{hours}
\DeclareSIUnit\minutes{minutes}

% Pienta's defines:
\newcommand{\REM}[1]{}
\DeclareMathOperator*{\argmax}{\arg\!\max}


\newcommand{\MakeAcronym}[3]{
  \acrodef{#2}[{\scshape #2}]{#3}
  \newcommand{#1}{\ac{#2}\xspace}
}
\MakeAcronym{\ourmethod}{GraSP}{Graph Streaming Partitioner}

\newcommand{\todo}[1]{\textcolor{red}{[#1]}}
\newcommand{\robert}[1]{\textcolor{OliveGreen}{[#1 -R]}}
\newcommand{\acar}[1]{\textcolor{blue}{[#1 -A]}}
\newcommand{\polo}[1]{\textcolor{red}{[#1 -P]}}
\newcommand{\alex}[1]{\textcolor{Mahogany}{[#1 - A]}}

\newcommand{\xourname}{Grasp}
\newcommand{\name}{\textsc{\xourname}\xspace}
%\newcommand{\ourmethod}{\name}
\newcommand{\ourmethodtitle}{Grasp\xspace}

\newcommand{\expparm}{\ensuremath{t_\alpha}\xspace}
\newcommand{\numrestreamsparm}{\ensuremath{n_s}\xspace}
\newcommand{\numprocsparm}{\ensuremath{p}\xspace}


\newcommand{\largestgraphedges}{34.3 billion\xspace}
\newcommand{\largestgraphtime}{15 seconds}

\newcommand{\ignore}[1]{}


\begin{document}
%
% --- Author Metadata here ---
\conferenceinfo{KDD}{'15 Sydney, Australia}
%\CopyrightYear{2007} % Allows default copyright year (20XX) to be over-ridden - IF NEED BE.
%\crdata{0-12345-67-8/90/01}  % Allows default copyright data (0-89791-88-6/97/05) to be over-ridden - IF NEED BE.
% --- End of Author Metadata ---

\title{GraSP: Distributed Streaming Graph Partitioning}
%\subtitle{[Extended Abstract]
%\titlenote{A full version of this paper is available as
%\textit{Author's Guide to Preparing ACM SIG Proceedings Using
%\LaTeX$2_\epsilon$\ and BibTeX} at
%\texttt{www.acm.org/eaddress.htm}}}
%
% You need the command \numberofauthors to handle the 'placement
% and alignment' of the authors beneath the title.
%
% For aesthetic reasons, we recommend 'three authors at a time'
% i.e. three 'name/affiliation blocks' be placed beneath the title.
%
% NOTE: You are NOT restricted in how many 'rows' of
% "name/affiliations" may appear. We just ask that you restrict
% the number of 'columns' to three.
%
% Because of the available 'opening page real-estate'
% we ask you to refrain from putting more than six authors
% (two rows with three columns) beneath the article title.
% More than six makes the first-page appear very cluttered indeed.
%
% Use the \alignauthor commands to handle the names
% and affiliations for an 'aesthetic maximum' of six authors.
% Add names, affiliations, addresses for
% the seventh etc. author(s) as the argument for the
% \additionalauthors command.
% These 'additional authors' will be output/set for you
% without further effort on your part as the last section in
% the body of your article BEFORE References or any Appendices.

\numberofauthors{3} %  in this sample file, there are a *total*
% of EIGHT authors. SIX appear on the 'first-page' (for formatting
% reasons) and the remaining two appear in the \additionalauthors section.
%
\author{
% You can go ahead and credit any number of authors here,
% e.g. one 'row of three' or two rows (consisting of one row of three
% and a second row of one, two or three).
%
% The command \alignauthor (no curly braces needed) should
% precede each author name, affiliation/snail-mail address and
% e-mail address. Additionally, tag each line of
% affiliation/address with \affaddr, and tag the
% e-mail address with \email.
%
% 1st. author
\alignauthor Casey Battaglino\\
       \affaddr{Georgia Institute of Technology}\\
       \email{cjbattagl@gatech.edu}
% 2nd. author
\alignauthor Robert Pienta\\
       \affaddr{Georgia Institute of Technology}\\
       \email{pientars@gatech.edu}
% \alignauthor Guy Fieri\\
\alignauthor Richard Vuduc\\
       \affaddr{Georgia Institute of Technology}\\
       \email{richie@gatech.edu}
%
}
% There's nothing stopping you putting the seventh, eighth, etc.
% author on the opening page (as the 'third row') but we ask,
% for aesthetic reasons that you place these 'additional authors'
% in the \additional authors block, viz.
%\additionalauthors{Additional authors: John Smith (The Th{\o}rv{\"a}ld Group,
%email: {\texttt{jsmith@affiliation.org}}) and Julius P.~Kumquat
%(The Kumquat Consortium, email: {\texttt{jpkumquat@consortium.net}}).}
%\date{30 July 1999}
% Just remember to make sure that the TOTAL number of authors
% is the number that will appear on the first page PLUS the
% number that will appear in the \additionalauthors section.

\maketitle
\begin{abstract}
Graph partitioning is the process of dividing a graph into subgraphs such that the number of edges between subgraphs is reduced. 
Our project investigates the process of partitioning a graph using a \emph{streaming} partitioner, which must make partition decisions as each vertex is read from memory, simulating an online algorithm that must process nodes as they arrive. 
In this project, we demonstrate and analyze streaming graph partitioning on a wide spectrum of real-world and synthetic graphs, as well as explore ways of improving partition quality. 
\end{abstract}



% A category with the (minimum) three required fields
\category{G.2.2}{Mathematics of Computing}{Discrete Mathematics}[Graph Algorithms]
%A category including the fourth, optional field follows...
% \category{D.2.8}{Software Engineering}{Metrics}[complexity measures, performance measures]
\terms{Theory}

\keywords{graph partitioning, streaming graph partitioning}

%!TEX root=kdd15_workshop_main.tex
\section{Introduction}
\subsection{Graph Partitioning}
Graph partitioning is standard problem in theoretical computer science and has had increasing attention as a method of dividing up problems to improve distributed-algorithm performance.
By dividing up a graph into equally sized partitions, with minimal edges connecting the groups, very large graphs can be processed with excellent concurrency. 

We wish to partition the nodes of a graph into $k$ balanced components with capacity $(1+\epsilon)\frac{N}{k}$, such that the number of edges that cross partition boundaries is minimized.
The number of inter-partition edges, which we seek to minimize, is often called the \textit{cost}.
The process of balancing partition size while maintaining minimum cost can be reduced to the minimum-bisection problem~\cite{Garey:1979:CIG:578533} and is therefore NP-Complete.
For this reason it is computationally infeasible to expect an optimal solution for even modest-sized graphs; approximation techniques are common.

% scenario
An effective partitioning of a graph can greatly improve the performance of graph algorithms and even allow for massive graphs to be fit into distributed memory.
Consider a parallel Breadth-First Search (BFS) where a graph's vertices (vertex edge lists) are partitioned between two machines.
During each BFS step, each process must communicate all newly explored vertices to the owner of those vertices.
In Figure~\ref{fig:0}, if we have 4 processes, all 14 nonzeros in the non-diagonal blocks must be communicated at some point.
A good partitioner concentrates nonzeros in the diagonal blocks, thereby reducing communication (which is the main bottleneck in almost all graph computations).

Offline graph partitioning algorithms have existed for decades.
They work by storing the graph in memory with complete information about the edges.
Hundreds of variants of these algorithms exist and range from spatial methods~\cite{Gilbert95geometricmesh} to spectral methods~\cite{arora2009expander}.
Some of the most scalable and effective graph partitioners are multi-level partitioners, which recursively contract the graph to a small number of vertices, and then heuristically optimize the problem on each subsequent expansion~\cite{karypis1998multilevel}.

\paragraph{Why Use Streaming Partitioning?}
The most salient property of streaming partitioning is its speed: it can partition the graph in a single sweep, with $O(|E|)$ memory access, storage, and run time.
Existing graph partitioners require the whole graph to be represented in memory, whereas streaming graph partitioning can process vertices as they arrive.

For example, partitioning a 26GB Twitter graph can take nearly a day using the fastest offline algorithms, but can take a matter of minutes using a streaming algorithm, with similar partition quality~\cite{tsourakakis2012fennel}.
This also suggests that we could do multiple passes of a streaming partitioner (using the same or different heuristics) to further improve the partitioning, all in a fraction of the time that an offline partitioner would take to terminate.


\begin{figure}[h]
\centering
\includegraphics[width=0.85\columnwidth] {figures/graphpart1.png}
\caption[Caption for]{Graph 4-partition shown with corresponding adjacency matrix}
\label{fig:0}
\end{figure}


Streaming graph partitioning has gained traction in the last few years~\cite{DBLP:journals/corr/abs-1212-1121,Stanton:2012:SGP:2339530.2339722,tsourakakis2012fennel}.
As graphs grow to the point where they do not fit into memory or must be distributed to compute nodes on the fly, we need new methods that support partitioning on only limited information.
In the streaming model, input data (vertices) arrive sequentially from a generating source (such as a web-crawler), and must be partitioned as they arrive.

Streaming partitioning is dependent on the order in which vertices arrive.
A web crawler might generate vertices in an order that represents a Breadth-First or Depth-First traversal of the web, or we may even receive vertices in a random order.
An analysis of streaming algorithms may also consider an adversarial ordering that produces the worst possible results~\cite{Stanton:2012:SGP:2339530.2339722}.

We have created \ourmethod, a fast, iterative, distributed streaming graph partitioner.
It works by restreaming the graph with tempered partition parameters to achieve a fast, parallel \textit{k}-partitioning.
We make the following contributions:
\begin{itemize}
\item A \textbf{scalable} distributed partitioner implementation using MPI.
\item Support for \textbf{streaming} partitioning regardless of stream order.
\item An \textbf{iterative} approach that creates quality partitions.
\end{itemize}




%!TEX root=kdd15_workshop_main.tex
\section{Methodology}
While there are many possible heuristics for streaming partitioning~\cite{Stanton:2012:SGP:2339530.2339722}, the most effective by far have been \emph{weighted, greedy} approaches. We keep track of the partition assignments of vertices streamed so far ($P_i^t$ for each process $i$ at time $t$). As each vertex $v$ is streamed, we count the edges from that vertex to each partition $|P_i^t \cap N(v)|$. This intuitively maximizes \emph{modularity}, the ratio of intra-partition edges to inter-partition edges. However, using this value on its own would result in all vertices being assigned to a single, large partition. Thus, we exponentially \emph{weight} the edge counts by the size of partitions $|P_i^t|$, relatively dampening the scores for partitions that are too large (but penalizing only lightly for small differences in size). This gives us two parameters: the linear importance of partition size to the score, $\alpha$, and the exponential rate at which increasing partition size incurs a greater penalty, $\gamma$. This yields the basic `FENNEL' algorithm~\cite{tsourakakis2012fennel} shown in~\RefAlgorithm{alg:fennel}.

\begin{algorithm}
 Set all $P_i$ to $\emptyset$\;
 \ForEach{$v \in V(G)$ as it arrives at time $t$}{
   $j \gets \displaystyle \argmax_{i\in\{1,\dots,p\}}|P_i^t \cap N(u)| - \alpha \frac{\gamma}{2}|P_i^t|^{\gamma-1}$\;
   Add $v$ to set $P_j^{t+1}$\;
 }
 \caption{Serial streaming FENNEL partitioner}
 \label{alg:fennel}
\end{algorithm}

Exact computation of this algorithm as described is not possible in parallel, because $P_i^{t-1}$ must be known to compute $P_i^t$. A multi-threaded approximation of this algorithm is easily performed by relaxing this requirement and using $P_i^{t-p}$ to compute $P_i^t$, where $p$ is the number of threads. This resulted in only a small drop in partition quality in our experiments.

To compute this algorithm in distributed memory, a naive approach is to constantly broadcast partition assignments as they are computed. Unless we use synchronization (which involves a massive performance hit), this results in a drastic drop in partition quality because the latency across a network is high enough that partition assignments are perpetually out of date. 

Our implementation follows the methodology of `restreaming' partitioning'~\cite{nishimura2013restream}, which shows the single-pass algorithms of FENNEL and WDG~\cite{tsourakakis2012fennel,Stanton:2012:SGP:2339530.2339722} can be repeated over the same data in the same order, yielding a convergent improvement in quality. This approach has other benefits that we utilize:

\begin{itemize}
\item Partition data is only communicated between streams, yielding high parallelism.
\item Parameters ($\alpha, \gamma)$ can be `tempered' to achieve higher-quality, balanced results that avoid immediate global minima.
\end{itemize}

\subsection{\ourmethod}
\ourmethod operates on a distributed graph $G$ in distributed CSR format. We take as input the parameters $\alpha, \gamma$, the number of partitions \numprocsparm (assumed to be equal to the number of MPI processes), the number of re-streams $\numrestreamsparm$, and the `tempering' parameter \expparm. \ourmethod then performs $\numrestreamsparm$ iterative passes over the graph (in identical random order), multiplicatively increasing the balance parameter by \expparm with each pass. This promotes a high-quality but less-balanced partition early on, while further promoting balance with subsequent pass~\cite{nishimura2013restream}. 

In between each pass, the partition information is communicated across all processors using the MPI \textsc{AllGather} primitive, which is often optimized for a given network architecture. The pseudocode for \ourmethod is shown in~\RefAlgorithm{alg:grasp}.


\begin{algorithm}
\ForPar{each process $p$}{
	$vorder \gets rand\_perm(\{0,\dots,|V(G_{local})|\})$\;
	Randomly assign local vertices to partitions $P_{i,p}^0$\;
}
\For{$run \gets \{ 1 \dots \numrestreamsparm \}$} {
	\ForPar{each process $p$}{
		\ForEach{$v \in vorder$}{
			$k \gets \displaystyle \argmax_{i\in\{1,\dots,p\}}|P_{i,p}^t \cap N(u)| - \alpha \frac{\gamma}{2}|P_{i,p}^t|^{\gamma-1}$\;
			Add $v$ to set $P_{k,p}^{t+1}$\;
		}
	}
	\textsc{MPI\_AllGather} global partition assignments\;
}
 \caption{Parallel Restreaming performed by \ourmethod.}
 \label{alg:grasp}
\end{algorithm}

This method is illustrated graphically in~\RefFigure{fig:restream}. In practice, we store the partitioning in a single compressed array, updating partition assignments in-place while storing a running count of the partition sizes. To increase accuracy we occasionally recompute global partition sizes using the \textsc{MPI\_AllReduce} primitive. 

\begin{figure}[ht]
\centering
  \includegraphics[width=1.0\columnwidth]{figures/restreamdiagram.pdf}
  %\caption{Partition speed of various Kronecker graphs.}
  \label{fig:restream}
  \caption{Two parallel restreaming steps on four processes.}
\end{figure}

Each process computes $\bigO{\numrestreamsparm \cdot \frac{|E|+|V|}{p}}$ work, and the network incurs a time of $\numrestreamsparm \cdot T_{allgather}(|V(G)|)$. 

%!TEX root=kdd15_workshop_main.tex
\section{Evaluation}  \label{sec:eval}
We ran our distributed experiments on a subset of the Edison machine at NERSC, featuring 5576 compute nodes with two 12-core Intel ``Ivy Bridge'' processors per node and a Cray Aries interconnect. 

We evaluate \ourmethod by its runtime as well as the quality of the partition that it produces, which we measure with \textit{fraction of cut edges} $\lambda$.
\begin{align}\lambda = \frac{\text{Number of edges cut by partition}}{\text{Total number of edges}}\end{align} where lower numbers represent a higher degree of locality. We can compare this to our baseline, the expected quality of a random $k-$partition, $\lambda_r = \frac{k-1}{k}$. Any partitioner that produces partitions with $\lambda < \lambda_r$ has improved the parallel locality of the partitions.

Balance is also an important metric in partitioning. Our basic metric for balance is the number of vertices in the largest partition divided by the number of vertices in the smallest partition, and we design our restreaming framework to perform a tempered restream until balance is within a decent tolerance ($\approx 1.2$).

\subsection{Test Graphs}
We measure our approach with both synthetic and real-world graphs. While synthetic graphs make for excellent scalability experiments, demonstration on real-world networks is important to verify that the partitioner works well in practice. 

\subsubsection{Real-world Graphs}
The SNAP dataset is a collection of real-world networks collected by Leskovec and collaborators~\cite{Leskovec-data, snapnets}. 
Many networks in this collection are power-law and scale-free representatives of social networks (such as collaboration networks, citation networks, email networks, and web graphs). 
We consider these to be excellent representative networks for a variety of domains. It is these types of networks that will continue to increase in size in the years to come.
We ran \ourmethod on a representative selection of these graphs, and outline the results in~\RefTable{tab:rw} and in~\refsec{sec:qual}.

\begin{table}
\caption{Basic properties of graphs in SNAP data set~\cite{Leskovec-data}, and $\lambda$ for one pass. $\lambda_{r,2}=0.5,\lambda_{r,8}=0.87$}
\rowcolors{2}{blue!05}{blue!15}
\centering
\small
{ \begin{tabular}{ *5r }    \toprule
\emph{Data Set} & $N$ & $nnz$  & $\lambda_{p=2}$ & $\lambda_{p=8}$ \\\midrule
soc-LiveJournal & 4,847,571 & 68,993,773  &0.234& 0.463\\
as-Skitter & 1,696,415 & 22,190,596  & 0.166&0.324\\
cit-Patents & 3,774,768 & 16,518,948  & 0.402&0.726\\
roadNet-CA & 1,971,281 & 5,533,214  & 0.186&0.360\\
web-Google & 916,428 & 5,105,039  &0.189&0.336\\
wiki-Talk & 2,394,385 & 5,021,410 &0.411&0.752\\
amazon0302 & 262,111 & 1,234,877 & 0.202&0.370\\
soc-Slashdot0902 & 82,168 & 948,464  &0.236&0.382\\
ca-AstroPh & 18,772 & 396,160 & 0.232&0.413\\
cit-HepPh & 34,546 & 421,578 & 0.343&0.646\\
email-EuAll & 265,214 & 420,045 & 0.280&0.538\\
Oregon-1 & 11,492 & 46,818  & 0.224&0.406\\
p2p-Gnutella04 & 10,879 & 39,994  & 0.415&0.747\\
 \hline
\end{tabular}\par
}
\label{tab:rw}
\end{table}

\subsubsection{Synthetic Graphs}
For scalability experiments we generated random undirected power-law Kronecker (RMAT) graphs of varying scale in parallel using the Graph500 Reference implementation~\cite{graph500}. Kronecker graphs are commonly used in HPC graph benchmarks and testing, and arbitrarily large instances can be efficiently generated in parallel. The \emph{scale} of an RMAT graph is equal to $\log |V(G)|$, and the edge-factor is the average number of edges per node, which we hold constant at 16. Vertex and edge counts for the scales we experiment on are shown in~\RefTable{tab:rmat}.

\begin{table}
\caption{Edge and vertex counts for generated RMAT graphs of each scale.}
\rowcolors{2}{blue!05}{blue!15}
\centering
\small
{ \begin{tabular}{ l | c | c | c | c | c | c  }    \toprule
Scale & 26 & 27 & 28 & 29 & 30 & 31 \\ \midrule
|V(G)| & 67M & 134M & 268M & 537M & 1.07B & 2.15B \\%& 4.29B \\
|E(G)| & 1.07B & 2.14B & 4.29B & 8.58B & 17.1B & 34.3B \\%& 68.7B \\
\hline
\end{tabular}\par
}
\label{tab:rmat}
\end{table}

\subsection{Scalability}
\subsubsection{Weak Scaling}
Weak-scaling holds the amount of data per process constant as we increase the number of processes. In our experimental setup we achieve this by doubling the number of MPI processes every time we increase the scale of the RMAT generator. This yields the per-stream timing experiments in~\RefFigure{fig:kronspeed_weak}, where each line is labeled with the size of data per process:
\begin{figure}[t!]
\centering
  \includegraphics[width=0.9\columnwidth]{figures/weak_scaling.pdf}
  \caption{Per-stream times of \ourmethod in a weak-scaling experiment. }
  \label{fig:kronspeed_weak}
\end{figure}

\begin{table}
\caption{Weak scaling results for ParMETIS on RMAT graphs, with $2^{18}$ vertices per compute node.}
\rowcolors{2}{blue!05}{blue!15}
\centering
\small
{ \begin{tabular}{ l | c | c | c | c | c }    \toprule
\#procs & 8 & 16 & 32 & 64 & 128 \\ \midrule
Time (s) & 5.01 & 10.2 & 25.0 & 64.0 & 167.0 \\
\hline
\end{tabular}\par
}
\label{tab:rmatpmweak}
\end{table}

This demonstrates that, for a reasonable number of MPI processes, we can scale up our problem sizes without encountering wasteful overhead from the network.

\subsubsection{Strong Scaling}
In strong-scaling, the size of the data is fixed while the number of processes inreases. 
Strong-scaling is heavily penalized by serial portions of code (as dictated by Amdahl's law) and growing network overhead. \ourmethod exhibits a high degree of parallelism, illustrated in~\RefFigure{fig:kronspeed_strong}. 

\begin{figure}[b!]
\centering
  \includegraphics[width=0.9\columnwidth]{figures/strong_scaling.pdf}
  \caption{Per-stream times of \ourmethod for various strong-scaling data sizes. For instance, we can perform a single partitioning pass over a 34 billion edge, 2.1 billion node network in just 15 seconds.}
  \label{fig:kronspeed_strong}
\end{figure}

\todo{In~\RefTable{tab:rmatpmstrong} we show strong scaling times for two RMAT scales using ParMETIS.}

\begin{table}
\caption{Strong scaling results for ParMETIS on RMAT graphs.}
\rowcolors{2}{blue!05}{blue!15}
\centering
\small
{ \begin{tabular}{ l | c | c | c | c | c }    \toprule
\#procs & 8 & 16 & 32 & 64 & 128 \\ \midrule
Scale 20 time (s) & 34.8 & 30.4 & 25.0 & 24.0 & 24.3 \\%& 68.7B \\
Scale 20 $\lambda$ & 0.36 & 0.38 & 0.40 & 0.42 & 0.45 \\%& 68.7B \\
Scale 22 time (s) & 307.8 & 221.9 & 194.9 & 173.8 & 167.0 \\
Scale 22 $\lambda$ & 0.36 & 0.39 & 0.41 & 0.43 & 0.45 \\
\hline
\end{tabular}\par
}
\label{tab:rmatpmstrong}
\end{table}

Performance inevitably plateaus for \ourmethod as local problem sizes become small in the face of increasing network overhead. However, for smaller degrees of parallelism we demonstrate near-linear scaling. 

\subsection{Quality} \label{sec:qual}
In \RefTable{tab:rw} we show some properties of our real test-graphs, as well as the performance of our streaming partitioner on them, for $p=2$ and $p=8$ partitions.. 

We show the validity of the restreaming approach in Figs.~\ref{fig:k2_lambda, fig:k16_lambda}.

We also compare our partition quality with ParMETIS.
In~\RefFigure{tab:rmatpmstrong}, \ourmethod performs comparably with parMETIS. Streaming partitioning is a valid alternative to conventional offline approaches and can be integrated in distributed-memory, on-the-fly algorithms for big-data.

\begin{figure}[t!]
\centering
\includegraphics[width=0.9\columnwidth] {figures/real_k2_lambda.pdf}
\caption[Caption for]{Improvement in the edges cut ($\lambda$) over 5 passes for bi-partitions of each graph. Because there are only two partitions, the algorithm is able to quickly fix mistakes it made in the initial partitioning. Many of the errors made in the first pass are fixed in the second iteration, with diminishing improvement thereafter.}
\label{fig:k2_lambda}
\end{figure}

\begin{figure}[t!]
\centering
\includegraphics[width=0.9\columnwidth] {figures/real_k16_lambda.pdf}
\caption[Caption for]{Improvement in edges cut ($\lambda$) over 5 passes for $k=16$-partitions of each graph. Dividing the graph into 16 partitions makes the minimum edge cut problem much more challenging. Similar to the bi-partition results, we experience the best gain in the second pass and less in subsequent passes.}
\label{fig:k16_lambda}
\end{figure}

\subsection{Analysis}
Our scalability tests have demonstrated that \ourmethod is highly parallel and performs quality partitions far faster than more sophisticated algorithms. A single stream over a 34 billion edge, 2.1 billion node network can be done in just 15 seconds. Performing a constant number of restreams while tempering the balance parameter allows us to find a good tradeoff between partition balance and partition quality. 

Partitions of power-law graphs are known to involve such a tradeoff~\cite{Lang04findinggood}. Continuously better cuts can be found as we relax our requirements for vertex balance. This is illustrated in \todo{Table}, and opens up interesting algorithmic questions.

\begin{figure}[t!]
\centering
\includegraphics[width=0.9\columnwidth] {figures/tradeoff_process.pdf}
\caption[Caption for]{ \todo{WRITE THIS KC} }
\label{fig:process}
\end{figure}

\begin{figure}[t!]
\centering
\includegraphics[width=0.9\columnwidth] {figures/tradeoff_roc.pdf}
\caption[Caption for]{ \todo{WRITE THIS KC} }
\label{fig:tradeoff}
\end{figure}
% Other data from real-world was harder to analyze -- there are not enough wide differences between data sets' results to draw strong conclusions.
Despite the complexity of many of the real graphs, \ourmethod creates well-balanced partitions.
While Power-Law graphs are overwhelmingly considered to be difficult to partition~\cite{Abou-Rjeili:2006:MAP:1898953.1899055}, we have demonstrated that a very simple, fast, algorithm is capable of significantly reducing communication in their parallel computation. We also demonstrate in~\RefFigure{fig:k2_lambda} and \RefFigure{fig:k16_lambda} that additional passes can further reduce edges cut by up to a factor of 3. 

% Isolated comparisons we have made to the state-of-the-art partitioner METIS show that these results are competitive (usually within a factor of 2).
\REM{
One set of outlying (poorly-performing) data points are the Gnutella networks.
While Gnutella networks exhibit power-law-like topologies, elements of their algorithm truncate nodes from ever becoming extremely large. 
This heavy-clipping sets the Gnutella networks apart from many of the other social network topologies in our experiments.
The data set also has extremely low clustering coefficients and a very small number of closed triangles \cite{Ripeanu:2002:MGN:613352.613670}. 
Low clustering coefficients decrease the chance that neighbors of the current node-under-consideration share a partition. 
With more neighbors distributed across the partitions, many partitions will have roughly the same score, making optimal partition choices much harder.

A danger of naively running multiple passes is that one partition often becomes populated by very high-degree vertices. 
We attribute this to the ``dense core'' surrounded by a less-dense periphery that many scale-free graphs possess.
This can be observed qualitatively when scale-free graphs are embedded in a spectral space~\cite{Lang04findinggood}.

This dense partition tends to strongly emerge as we continue to make further passes of the streaming algorithm.
In order to overcome this we used the tempered parameter technique described in our methodology section. 
}

%!TEX root=kdd15_workshop_main.tex
\section{Related Work} \label{sec:rel}
Partitioning is an important step in many algorithms. In HPC applications ranging from simulation to web analytics, the quality of partitions can strongly affect the parallel performance of many algorithms. Partitioning can also be used to identify community structure. We mention here a small sample of contemporary work in graph partitioning.

Streaming partitioning for a variety of heuristics was first presented by Stanton and Kliot~\cite{Stanton:2012:SGP:2339530.2339722}, the Weighted Deterministic Greedy approach generalized by Tsourakakis, et. al~\cite{tsourakakis2012fennel}, and the benefits of restreaming for convergence and parallelism determined by Nishimura and Ugander~\cite{nishimura2013restream}, although large-scale parallel experiments and benchmarks were not performed. Our implementation is the first parallel HPC-oriented study that we are aware of. 

\paragraph{Large-scale Data-mining}
Shared-memory vs. Distributed argument: GraphCHI~\cite{graphchi}

\paragraph{Partitioning}
\REM{Spatial methods like \cite{Gilbert95geometricmesh} that leverage geometric data to supplement the partition formation process.
Arora et al. proposed a fast approximation algorithm using spectral projection to perform sparse cuts, edge expansions, and separator balance \cite{arora2009expander}.
Some of the most successful partitioners use multilevel approaches \cite{karypis1998multilevel}.
Which achieve a high level of concurrency while maintaining good partition quality.
Some recent approaches have been designed specifically to address small-world networks \cite{slota2014pulp}.}

For networks with dynamic structure, iterative approaches can dynamically adjust the partitions to suit changing graph structure.
Vaquero et al. propose a method for iteratively adjusting graph partitions to cope with changes in the graph, using only local information ~\cite{Vaquero:2013:APL:2523616.2525943}.
This work demonstrated the power and scalability of leveraging local data to improve partition quality, especially to reduce the edges cut.

Sedge or Self Evolving Distributed Graph Management Environment also takes advantage of dynamically managing and modifying partitions to reduce network communication and improve throughput~\cite{Yangpart}.

Frameworks like Pregel~\cite{Malpregel}, make use of hashing-based partition schemes.
These allow constant-time lookup and prediction of partition location based on only the vertex ids.
GraphLab~\cite{Low:2012:DGF:2212351.2212354} also uses a hashed, random partitioning method, which essentially produces a worst-case edgecut of size $\frac{k-1}{k}|E|$, but which has the benefit that $H(v)$ can be called at any time to return the compute node that owns $v$.  
Khayyat et al. showed that a preprocessed partitioning of large-scale graphs is insufficient to truly minimize network communication~\cite{khayyatmizan}.
They propose another dynamic partition approach that allows vertex migration during runtime to maintain balanced load.

Boman, et. al show how graph partitioning can be used to optimize distributed SpmV~\cite{Bomansc13}, but use more sophisticated partitioners. A streaming approach would be ideal in extending the scale of experiments such as this. 
Streaming partitioning has also been successfully adapted for edge-centric partitioning schemes like X-Stream~\cite{xstream}.
X-Stream uses edge partitioning, to streams edges rather than vertices, which takes advantage of increased sequential memory access bandwidth.





%!TEX root=kdd15_workshop_main.tex
\section{Conclusion} \label{sec:conc}
In this work, we demonstrated \ourmethod, a \emph{distributed, streaming} partitioner.
\ourmethod uses an efficient distributed, sparse representation of the graph to perform fast, scalable partitioning.
We demonstrated \ourmethod on real-world and synthetic graphs by analyzing both (1) partition quality and (2) implementation scalability.
\ourmethod is scalable and can partition a graph of \largestgraphedges edges in \largestgraphtime, while maintaining partition quality comparable with the competitors.
\ourmethod will allow researchers and analysts to generate partitions for their work faster and easier than before.




%\subsection{References}

\bibliographystyle{abbrv}
\bibliography{bibly}
\balancecolumns
% \thebibliography
 % GM June 2007
\end{document}

