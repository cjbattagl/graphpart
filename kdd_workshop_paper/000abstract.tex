%!TEX root=kdd15_workshop_main.tex
\begin{abstract}
The size of real-world graph data sets continues to grow past the limits of commodity machines. In distributing large-scale graph mining approaches, a first step is often partitioning the data across machines in a way that reduces communication volume while simultaneously promoting load-balance. 
While graph partitioning is a mature field, recent work has shown that low-overhead streaming heuristics can perform competitively with heavyweight, offline graph partitioners for many real-world graphs. 

In this work we investigate the process of partitioning a social graph using a \emph{distributed, streaming} partitioner, \ourmethod, which makes partition decisions as each vertex is read from memory, simulating an online algorithm that must process nodes as they arrive. 
We make use of the MPI paradigm, ubiquitous in the HPC world, to promote \ourmethod as a simple library to substitute for existing libraries such as parMetis. 

\ourmethod is the first MPI implementation of streaming partitioning that we are aware of, and demonstrates performance far exceeding existing partitioners while providing comparable results. We demonstrate the scalability of \ourmethod on up to 1024 compute nodes of NERSC's Edison supercomputer, with speedups of up to XX compared to a multilevel partitioner. 
\end{abstract}

