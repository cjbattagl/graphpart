%!TEX root=kdd15_workshop_main.tex
\begin{abstract}
Modern graph data have grown beyond the memory limits of most single machines.
The first the step in dividing up large graph problems is often partitioning the graph into manageable subgraphs.
The goal of partitioning is to both plit up the graph and minimize the number of edges between the subgraphs.
In this work we investigate the process of partitioning a graph using a \emph{distributed, streaming} partitioner, \ourmethod, which must make partition decisions as each vertex is read from memory, simulating an online algorithm that must process nodes as they arrive.
\ourmethod uses an efficient distributed, sparse representation of the graph to perform fast, scalable partitioning.
We demonstrate and analyze \ourmethod on a wide spectrum of real-world and synthetic graphs, as well as demnostrate several ways of improving partition quality.
\end{abstract}

