%!TEX root=kdd15_workshop_main.tex
\begin{abstract}

The size of real-world graph data sets continues to grow past the limits of commodity machines. An initial step in many large-scale graph mining approaches is partitioning the graph across machines in a way that reduces communication volume while simultaneously promoting load-balance. While graph partitioning is a mature field, recent work has shown that low-overhead streaming heuristics can perform competitively with heavyweight, offline graph partitioners when the data is scale-free. 

In this work we investigate the process of partitioning a social graph using a \emph{distributed, streaming} partitioner, \ourmethod, makes partition decisions as each vertex is read from memory, simulating an online algorithm that must process nodes as they arrive. We make use of the MPI paradigm, ubiquitous in the HPC world, to promote \ourmethod as a simple library to substitute for existing libraries such as parMetis. 

\ourmethod is the first MPI implementation of streaming partitioning that we are aware of. 
We demonstrate the scalability of \ourmethod on XXXX nodes on NERSC's Edison supercomputer on a wide spectrum of real-world and synthetic graphs, and demonstrate several ways of improving partition quality. 
\end{abstract}

